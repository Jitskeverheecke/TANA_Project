\chapter{Introduction}
\section{Cosmology}
\subsection{Cosmological parameters}
\begin{equation}
1 + z = \frac{1}{a}
\end{equation}
\begin{equation}
\Omega_m (z) = \frac{\Omega_{m,0} (1+z)^3}{\Omega_{m,0} (1+z)^3 + \Omega_{\Lambda,0}}
\end{equation}

\subsection{Unit system}
\begin{equation}
    kpc, M_\odot, Gyr
\end{equation}
A general function can be expanded in spherical harmonics as
\begin{equation}
    f(\textbf{r}) = f(r, \theta, \phi) = \sum_{l=0}^{\infty} \sum_{m=-l}^{l} f_{lm}(r) Y_{lm}(\theta, \phi)
\end{equation}
The coefficients are given by
\begin{equation}
    f_{lm}(r) = \int Y_{lm}^*(\theta, \phi) f(r, \theta, \phi) d\Omega
\end{equation}
where $d\Omega = \sin{\theta} d\theta d\phi$ is the solid angle element following the orthonormality relation of the spherical basis functions. For a fully spherically symmetric system, only the monopole term in the expansion survives, i.e. \(l=0, m=0\).

For a discrete sampling of the underlying function, the above expression amounts to a summation over all particles in a shell of radius $r_j$.
\begin{equation}
    f_{lm}(r_j) = \sum_{i} Y_{lm}^*(\theta_i, \phi_i) f(r_i, \theta_i, \phi_i)
\end{equation}
For a discrete distribution of particles, with $\rho(\textbf{r}_j) = \sum_{i} \delta(\textbf{r}_j - \textbf{r}_i)$, the coefficients can be expressed as
\begin{equation}
    \rho_{lm}(r_j) = \int Y_{lm}^*(\theta, \phi) \rho(r_j, \theta, \phi) d\Omega
\end{equation}
